%%%%%%%%%%%%%%%%%%%%%%%%%%%%%%%%%%%%%%%%%%%%%%%%%%%%
% mechTikz - Rúd szerkezetek rajzolása TikZ segítségével
%%%%%%%%%%%%%%%%%%%%%%%%%%%%%%%%%%%%%%%%%%%%%%%%%%%%

% --------------------------------------------------
% Alapstílusok
% --------------------------------------------------
% Definiáljuk a gyakran használt TikZ stílusokat
% (pl. pontok, rudak, erővektorok, stb.)

\tikzset{
  point/.style = {circle,fill=black,inner sep = 1.5pt},
  link/.style = {color=#1, very thick},
  angle/.style={color=#1,<->, >=latex},
  vec/.style = {color=#1,very thick,-latex},
  arc vec/.style={color=#1,very thick, postaction={decorate}, decoration={markings,mark=at position 1 with {\arrow{>}}}},
  arc vec2/.style={color=#1,very thick, postaction={decorate}, decoration={markings,mark=at position 1 with {\arrow{<}}}},
  force/.style = {color=#1, very thick,-latex},
  moment/.style = {color=#1,very thick,-Implies, double distance=1pt},
  coordSys/.style = {color=#1, thin,-latex},
  guide/.style = {color=#1,thin},
  guideVec/.style = {color=#1,latex-latex},
  construct/.style = {color=#1,thick,dashed},
  support/.style={link={black}, fill=white}
}

%%%%%%%%%%%%%%%%%%%%%%%%%%%%%%%%%%%%%%%%%%%%%%%%%%%%
% Saját TikZ parancsok definiálása
%%%%%%%%%%%%%%%%%%%%%%%%%%%%%%%%%%%%%%%%%%%%%%%%%%%%
% Ezek a parancsok új "rajzelemeket" hoznak létre, 
% amiket később egyszerűen, rövid paranccsal hívhatunk 
% meg rajzolás közben (pl. \supporthinge{A}{0}).

% --------------------------------------------------
% Általános elemek
% --------------------------------------------------

% Távolságjelölő makró
\newcommand{\dimarrow}[7][]{%
  % #1: opcionális TikZ stílus (pl. color=red, thick, stb.)
  % #2: első pont
  % #3: második pont
  % #4: eltolás vektor 1 (pl. (0,-1.25) vagy (1,0))
  % #5: eltolás vektor 2 (pl. (0,-1.25) vagy (1,0))
  % #6: felirat a nyíl közepén
  % #7: felirat pozíció (pl. below, above, left, right)
  \begin{scope}[#1]
    % vezető vonalak
    \draw[guide={gray}] ($(#2)+#4$) -- (#2);
    \draw[guide={gray}] ($(#3)+#5$) -- (#3);
    % távolság nyíl felirattal
    \draw[guideVec={gray}] ($(#2)+#4$) -- node[midway, #7] {#6} ($(#3)+#5$);
  \end{scope}
}

% Kis koordinátarendszer megrajzolása
\newcommand{\coordSys}[2][]{%
  % #1: opcionális paraméter (pl. scale=1.2, draw=blue, stb.)
  % #2: csomópont koordinátája (pl. (2,3))
  \begin{scope}[shift={(#2)}, #1]
    % x tengely
    \draw[coordSys={red}] (-0.5,0) -- (0.5,0) node[right] {$x$};
    % y tengely
    \draw[coordSys={red}] (0,-0.5) -- (0,0.5) node[above] {$y$};
  \end{scope}
}

% --------------------------------------------------
% Mechanikai elemek
% --------------------------------------------------

% Csuklós alátámasztás + ground
\newcommand{\supporthinge}[3][]{%
  % #1: opcionális paraméter (pl. scale=1.2, draw=blue, stb.)
  % #2: csomópont koordinátája
  % #3: forgatási szög fokban
  \begin{scope}[shift={(#2)}, rotate=#3, #1]
    % Talaj
    \fill[gray!20!white] (-0.5,-0.5) rectangle (0.5,-0.75);
    \draw[black, very thick] (-0.5,-0.5) -- (0.5,-0.5);
    % Háromszög
    \draw[support] (0,0) -- (-0.25,-0.5) -- (0.25,-0.5) -- cycle;
    \fill[white, draw={black}] (0,0) circle (0.1cm);
  \end{scope}
}

% Görgős alátámasztás + ground
\newcommand{\supportroller}[3][]{%
  % #1: opcionális paraméter
  % #2: csomópont koordinátája
  % #3: forgatási szög fokban
  \begin{scope}[shift={(#2)}, rotate=#3, #1]
    % Talaj
    \fill[gray!20!white] (-0.5,-0.5) rectangle (0.5,-0.75);
    \draw[black, very thick] (-0.5,-0.5) -- (0.5,-0.5);
    % Háromszög + görgők
    \draw[support] (0,0) -- (-0.25,-0.35) -- (0.25,-0.35) -- cycle;
    \foreach \i in {-0.15,0,0.15} {
      \draw[support] (\i,-0.425) circle (0.075);
    }
    \fill[white, draw={black}] (0,0) circle (0.1cm);
  \end{scope}
}

% Befogás
\newcommand{\supportfixed}[3][]{%
  % #1: opcionális paraméter (pl. line width, color, stb.)
  % #2: csomópont koordinátája
  % #3: forgatási szög fokban
  \begin{scope}[shift={(#2)}, rotate=#3, #1]
    % Talaj (befogás jelölés)
    \fill[gray!20!white] (-0.5,0) rectangle (0.5,-0.5);
    \draw[black, very thick] (-0.5,0) -- (0.5,0);
  \end{scope}
}

% Csúszka
\newcommand{\supportslider}[4][]{%
  % #1: opcionális paraméter (pl. line width, color, stb.)
  % #2: csomópont koordinátája
  % #3: forgatási szög fokban
  % #4: sín szélessége (alapértelmezett: 1)
  \begin{scope}[shift={(#2)}, rotate=#3, #1]
    % Felső sín
    \fill[gray!20!white] (-0.75,#4/2) rectangle (0.75,#4/2+0.25);
    \draw[black, very thick] (-0.75,#4/2) -- (0.75,#4/2);
    % Alsó sín
    \fill[gray!20!white] (-0.75,-#4/2) rectangle (0.75,-#4/2-0.25);
    \draw[black, very thick] (-0.75,-#4/2) -- (0.75,-#4/2);
  \end{scope}
}

% Kötél kényszer
\newcommand{\supportrope}[4][]{%
  % #1: opcionális paraméter (pl. scale, color, stb.)
  % #2: csomópont koordinátája
  % #3: forgatási szög fokban (pl. ha nem függőleges kábelt akarsz)
  \begin{scope}[shift={(#2)}, rotate=#3, #1]
    % Talaj (felfogatási pont)
    \fill[gray!20!white] (-0.5,-#4-0.25) rectangle (0.5,-#4-0.5);
    \draw[black, very thick] (-0.5,-#4-0.25) -- (0.5,-#4-0.25); 
    % Kötél
    \draw[black, thick, densely dashed] (0,0) -- (0,-#4-0.25);
    % Pontok
    \fill[black] (0,-#4-0.25) circle (0.05); % alsó pont
    \fill[black] (0,0) circle (0.05);        % felső pont
  \end{scope}
}

% Sima támasz + ground
\newcommand{\supportsimple}[3][]{%
  % #1: opcionális paraméter (pl. scale=1.2, draw=blue, stb.)
  % #2: csomópont koordinátája
  % #3: forgatási szög fokban
  \begin{scope}[shift={(#2)}, rotate=#3, #1]
    % Talaj
    \fill[gray!20!white] (-0.5,-0.5) rectangle (0.5,-0.75);
    \draw[black, very thick] (-0.5,-0.5) -- (0.5,-0.5);
    % Félkör támasz
    \draw[support] (0.25,-0.25) arc[start angle=0, end angle=180, radius=0.25];
    % Oldalsó vonalak a félkörhöz
    \draw[support] (-0.25,-0.25) -- (-0.25,-0.5) (0.25,-0.5) -- (0.25,-0.25) -- cycle;
  \end{scope}
}

% Csukló
\newcommand{\hinge}[2][]{%
  % #1: opcionális paraméter (pl. scale=1.2, draw=blue, stb.)
  % #2: csomópont koordinátája
  \begin{scope}[shift={(#2)}, #1]
    \fill[white, draw={black}] (0,0) circle (0.1cm);
  \end{scope}
}